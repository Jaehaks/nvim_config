% \documentclass[../main.tex]{subfiles}

% \begin{document}

\chapter{Title of Chapter 1}

\section{User guide}

%%%%%%%%%%%%%%%%%%%%%%%%%%%%%%%%%%
% equation writing
%%%%%%%%%%%%%%%%%%%%%%%%%%%%%%%%%%
\begin{align}
	PV = nRT \\
	PV = nRT
\end{align}

%%%%%%%%%%%%%%%%%%%%%%%%%%%%%%%%%%
% Write reference
%%%%%%%%%%%%%%%%%%%%%%%%%%%%%%%%%%


If you set `@maintex` as argument of latex.args,
you can execute compile function for main.tex file
even though you focus to subfile(chapter1.tex).
It detects main file automatically. read README.md


I set `lualatex` as pdflatex option so compile speed is slow.

it is test
it is test
it is test
it is test

this is continuous compile mode using inherit property of
latex engine.

\begin{align}
	PV = nRT \\
	PV = nRT
\end{align}

\begin{align}
	PV = nRT \\
	PV = nRTa
\end{align}

\begin{align}
	PV = nRT \\
	PV = nRT
\end{align}

\begin{align}
	PV = nRT \\
	PV = nRT
\end{align}


% cite \cite{bocker2007state}
% \cite{einstein1905zur}
\cite{f2020learn}
\cite{f2024learn}
\cite{f2024learn}
\cite{f2024learn}
\cite{f2024learn}

test
\cite{lamport1994latex}

\bibliographystyle{plain}
% \bibliography{chapters/ch1bib.bib}
\bibliography{chapters/ch1bib}


% \end{document}
