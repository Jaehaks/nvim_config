% lualatex 기준

% Latex Guide
% 1) eqnarray 미사용
%	- 등호 좌우간격이 너무 넓어서 보기 안좋음
%	- 수식이 수식 번호까지 겹치게 되는 문제
%	- 여러줄 수식을 조판하기 위한 환경인데, 이득이 없고 `align`을 대신 사용, align*는 수식 번호 없음
%	- 여러줄 수식에 수식 번호 하나 붙이려면 `align`이나 `equation` block 안에  `aligned` 이용


% =====================================================================
% Document Start / layout
% =====================================================================
% documentclass : prefix 로 ext사용시(extreport) 13pt 이상 폰트 크기 사용 가능
%				  chapter / section 등은 book, report class 를 이용해야 한다
% fleqn : 작성되는 수식들을 왼쪽으로 정렬, 없으면 가운데 정렬
% font size : 일반적으로 12pt가 한계
\documentclass[12pt]{report}



% - geometry : 여백 조정 (기본 여백이 너무 많음)
% top / bottom 은 chapter / section 등이 표시되기 위한 header / footer 공간은 남겨야한다
% \chapter{} block 은 원래 위쪽에 마진이 많다. chapter 표시는 페이지 맨 위에 header 에 표시
% landscape : 가로방향 페이지
\usepackage[left=1cm, right=1cm, top=2cm, bottom=2cm, landscape]{geometry}


% - setspace : 줄간 간격 조정
% (singlespacing 으로 라인간 한칸 유지) : 이게 없으면 내용에 따라 latex가 알아서 자동 줄간격 조정할 수 있다
\usepackage{setspace}
\singlespacing

% - indentfirst : 자동 첫 줄 들여쓰기
% 				\setlength{\parindent}{0.2in} : 들여쓰기 간격 설정
%				\setlength{\parskip}{6pt} : 문단 간 간격 조정
\usepackage{indentfirst}



% =====================================================================
% Encoding 및 언어 지원
% =====================================================================
% \usepackage[utf8]{inputenc} % inputenc : latex 문서 text encoding utf8로 설정 (lualtex에서는 무시됨) pdflatex용
\usepackage[T1]{fontenc} % [T1] 은 lualtex에서 무시됨, pdflatex용
\usepackage{kotex} % [T1] 은 lualtex에서 무시됨, pdflatex용

% =====================================================================
% 참조
% =====================================================================
\usepackage{url}
\usepackage[sectionbib]{chapterbib}
% \usepackage{subfiles} % sub file구조로 사용 가능
% \usepackage{cite}
% \usepackage[backend=biber, style=ieee]{biblatex}
% \addbibresource{chapters/ch1-bib.bib}
% \addbibresource{chapters/chapter2.bib}

% =====================================================================
% 수학 기호
% =====================================================================
\usepackage{amsmath} % matrix / align / dfrac block등을 사용 가능 , (이게 없으면 array block 으로만 만들어야 한다)
% \usepackage{amssymb} % blackboard/ relational / logical 등 추가적인 수학기호 사용 가능
% \usepackage{amsfonts} % fraktur/ calligraphic 등 여러 수학기호 폰트
% \usepackage{bm} % bold math symbols
% \usepackage{physics} % 물리수식, 연산자 사용



% =====================================================================
% 추가 유틸
% =====================================================================
% \usepackage{graphicx}
% \usepackage{booktabs}
% \usepackage{hyperref} % \autoref, \nameref 사용 가능, hyperlink 자동 생성
% \hypersetup{
% 	colorlinks=true,
% 	linkcolor=black,
% 	citecolor=blue,
% 	urlcolor=cyan,
% }
% \usepackage{cleveref} % \cref, \Cref이용, hyperref다음에 로드

% =====================================================================
% 제목 정보 (\maketitle 페이지에 들어가는 정보)
% =====================================================================
% {book} class의 경우, 양면 기준이기 때문에, 짝수/홀수 페이지에 있어야 하는 내용 기준이 엄격함
% 이게 싪다면 {report} class로 하면 된다
\title{title of documents}
\author{KimJaehak}
\date{\today}


%%%%%%%%%%%%%%%%%%%%%%%%%%%%%%%%%%%%%%%%%%%%%%%%%%%%%%%%%%%%%%%%%%%%%%%%%%%%%%%%%%%%%%%%%%%%%%%%%%%%%%%%%%%%
%% document start
%%%%%%%%%%%%%%%%%%%%%%%%%%%%%%%%%%%%%%%%%%%%%%%%%%%%%%%%%%%%%%%%%%%%%%%%%%%%%%%%%%%%%%%%%%%%%%%%%%%%%%%%%%%%
\begin{document}

\maketitle % add title page
\tableofcontents % add TOC page

\newpage
% \documentclass[../main.tex]{subfiles}

% \begin{document}

\chapter{Title of Chapter 1}

\section{User guide}

%%%%%%%%%%%%%%%%%%%%%%%%%%%%%%%%%%
% equation writing
%%%%%%%%%%%%%%%%%%%%%%%%%%%%%%%%%%
\begin{align}
	PV = nRT \\
	PV = nRT
\end{align}

%%%%%%%%%%%%%%%%%%%%%%%%%%%%%%%%%%
% Write reference
%%%%%%%%%%%%%%%%%%%%%%%%%%%%%%%%%%


If you set `@maintex` as argument of latex.args,
you can execute compile function for main.tex file
even though you focus to subfile(chapter1.tex).
It detects main file automatically. read README.md


I set `lualatex` as pdflatex option so compile speed is slow.

it is test
it is test
it is test
it is test

this is continuous compile mode using inherit property of
latex engine.

\begin{align}
	PV = nRT \\
	PV = nRT
\end{align}

\begin{align}
	PV = nRT \\
	PV = nRTa
\end{align}

\begin{align}
	PV = nRT \\
	PV = nRT
\end{align}

\begin{align}
	PV = nRT \\
	PV = nRT
\end{align}


% cite \cite{bocker2007state}
% \cite{einstein1905zur}
\cite{f2020learn}
\cite{f2024learn}
\cite{f2024learn}
\cite{f2024learn}
\cite{f2024learn}

test
\cite{lamport1994latex}

\bibliographystyle{plain}
% \bibliography{chapters/ch1bib.bib}
\bibliography{chapters/ch1bib}


% \end{document}


\newpage
%!TEX root = ../main.tex
% \documentclass[../main.tex]{subfiles}

% \begin{document}

\chapter{This is title of chapter 2}
\section{Introduction to Chapter 2}

% 반복 텍스트로 200줄 채움

\begin{equation}
F = ma
\end{equation}

test
otest
testes
testes
testes
testes
testes
testes

you can edit any file in project, but main.tex will be compiled
It doesn't need to continuous mode. if you select @maintex as argument

check it is test
check it is test
check it is test
check it is test
\begin{align}
y = mx + b \\
\sin^2 \theta + \cos^2 \theta &= 1
\end{align}

% 1. 덧셈, 뺄셈, 곱셈, 나눗셈
\( A + B - C \times D \div E \)

% 2. 분수
\[
\frac{a}{b}
\]

% 3. 합 (시그마)
\[
\sum_{i=0}^n i
\]

% 4. 적분
\[
\int_{0}^{1} x^2 \, dx
\]

% 5. 제곱근
\[
\sqrt{a^2 + b^2}
\]

% 6. 알파, 베타 등 그리스 문자
\[
\alpha + \beta = \gamma
\]

% 7. 부등호와 관계 기호
\[
a \leq b, \quad x \neq y
\]


% 7. 부등호와 관계 기호
\[
a \leq b, \quad x \neq y
\]

% 8. 부분집합과 포함 기호
\[
A \subseteq B, \quad x \in A
\]

% 9. 편미분 기호
\[
\frac{\partial f}{\partial x}
\]

% 10. 화살표
\[
x \rightarrow y
\]

\cite{bocker2007state}

\bibliographystyle{plain}
% \bibliography{chapters/ch1bib.bib}
\bibliography{chapters/chapter2}

% \end{document}


\newpage
% \documentclass[../main.tex]{subfiles}

% \begin{document}

\chapter{This is title of chapter 3}

\section{Introduction to Chapter 3}

% 200줄 채우기 (예시: 반복 텍스트와 수식)

\begin{equation}
E = mc^2
\end{equation}


\begin{align}
a + b &= c \\
x^2 + y^2 &= z^2
\end{align}




% \end{document}




\end{document}
