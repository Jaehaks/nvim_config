% Latex Guide
% 1) eqnarray 미사용
%	- 등호 좌우간격이 너무 넓어서 보기 안좋음
%	- 수식이 수식 번호까지 겹치게 되는 문제
%	- 여러줄 수식을 조판하기 위한 환경인데, 이득이 없고 `align`을 대신 사용, align*는 수식 번호 없음
%	- 여러줄 수식에 수식 번호 하나 붙이려면 `align`이나 `equation` block 안에  `aligned` 이용


% =====================================================================
% Document Start / layout
% =====================================================================
% documentclass : article 대신 extarticle class 이용 시 13pt 이상 폰트 크기 사용 가능
%				  단, chapter / section 등은 사용 불가 , 이를 이용하려면 book, report class 를 이용해야 한다
% fleqn : 작성되는 수식들을 왼쪽으로 정렬, 없으면 가운데 정렬
% font size : 일반적으로 12pt가 한계
\documentclass[12pt, a4paper]{report}

% - geometry : 여백 조정 (기본 여백이 너무 많음)
% top / bottom 은 chapter / section 등이 표시되기 위한 header / footer 공간은 남겨야한다
% \chapter{} block 은 원래 위쪽에 마진이 많다. chapter 표시는 페이지 맨 위에 header 에 표시
% landscape : 가로방향 페이지
\usepackage[left=1cm, right=1cm, top=2cm, bottom=2cm, landscape]{geometry}
% - setspace : 줄간 간격 조정
% (singlespacing 으로 라인간 한칸 유지) : 이게 없으면 내용에 따라 latex가 알아서 자동 줄간격 조정할 수 있다
\usepackage{setspace}
\singlespacing
% - indentfirst : 자동 첫 줄 들여쓰기
% 				\setlength{\parindent}{0.2in} : 들여쓰기 간격 설정
%				\setlength{\parskip}{6pt} : 문단 간 간격 조정
\usepackage{indentfirst}

% =====================================================================
% Encoding 및 언어 지원
% =====================================================================
% - inputenc : latex 문서 text encoding utf8로 설정
\usepackage[utf8]{inputenc}
% - fontenc : font encoding 을 Western European language 로 설정 (default OT1)
\usepackage[T1]{fontenc}
% - kotex : 한글 지원
\usepackage{kotex}

% =====================================================================
% 수학 기호
% =====================================================================
\usepackage{amsmath} % matrix / align / dfrac block등을 사용 가능 , (이게 없으면 array block 으로만 만들어야 한다)
\usepackage{amssymb} % blackboard/ relational / logical 등 추가적인 수학기호 사용 가능
\usepackage{amsfonts} % fraktur/ calligraphic 등 여러 수학기호 폰트
\usepackage{bm} % bold math symbols

% =====================================================================
% 참조
% =====================================================================
% \usepackage{cite}
% \usepackage[backend=biber]{biblatex}
% \addbibresource{Latex_Guide_bibtex}

% =====================================================================
% 추가 유틸
% =====================================================================
\usepackage{graphicx}
\usepackage{booktabs}
\usepackage{hyperref} % \autoref, \nameref 사용 가능, hyperlink 자동 생성
\hypersetup{
	colorlinks=true,
	linkcolor=black,
	citecolor=blue,
	urlcolor=cyan,
}
\usepackage{cleveref} % \cref, \Cref이용, hyperref다음에 로드

% =====================================================================
% 제목 정보 (\maketitle 페이지에 들어가는 정보)
% =====================================================================
% {book} class의 경우, 양면 기준이기 때문에, 짝수/홀수 페이지에 있어야 하는 내용 기준이 엄격함
% 이게 싪다면 {report} class로 하면 된다
\title{title of documents}
\author{KimJaehak}
\date{\today}


%%%%%%%%%%%%%%%%%%%%%%%%%%%%%%%%%%%%%%%%%%%%%%%%%%%%%%%%%%%%%%%%%%%%%%%%%%%%%%%%%%%%%%%%%%%%%%%%%%%%%%%%%%%%
%% document start
%%%%%%%%%%%%%%%%%%%%%%%%%%%%%%%%%%%%%%%%%%%%%%%%%%%%%%%%%%%%%%%%%%%%%%%%%%%%%%%%%%%%%%%%%%%%%%%%%%%%%%%%%%%%
\begin{document}

\maketitle % add title page
\tableofcontents % add TOC page

\chapter{Induction Motor}

\section{Induction Motor Control and Dynamics}

%%%%%%%%%%%%%%%%%%%%%%%%%%%%%%%%%%%%
%% subsections
%%%%%%%%%%%%%%%%%%%%%%%%%%%%%%%%%%%%
\subsection{easy matrix}

It is test line
\begin{align}
	\label{eq:test}
	\begin{bmatrix}
		1 & 1 \\
		0 & 0 \\
	\end{bmatrix}
\end{align}

\subsection{induction motor voltage equation (complex vector)}
- induction motor voltage equation (complex vector)
\begin{align}
	\begin{aligned}
		% \label{eq:complex_stator}
		\underline{v}_s &= R_s \underline{i}_s + \frac{d\underline{\lambda}_s}{dt} + j \omega_e \underline{\lambda}_s \\
		% \label{eq:complex_rotor}
		\underline{v}_r &= R_r \underline{i}_r + \frac{d\underline{\lambda}_r}{dt} + j (\omega_e - \omega_r) \underline{\lambda}_r
	\end{aligned}
\end{align}


\begin{align}
	\begin{bmatrix}
		v_{qs}\\
		v_{ds}\\
		v_{qr}\\
		v_{dr}
	\end{bmatrix}
	&=
	\begin{bmatrix}
		R_s & -\omega_e L_s & 0 & -\omega_e L_m\\
		\omega_e L_s & R_s & \omega_e L_m & 0\\
		0 & -(\omega_e - \omega_r) L_m & R_r & -(\omega_e - \omega_r) L_r\\
		(\omega_e - \omega_r) L_m & 0 & (\omega_e - \omega_r) L_r & R_r
	\end{bmatrix}
	\begin{bmatrix}
		i_{qs}\\
		i_{ds}\\
		i_{qr}\\
		i_{dr}
	\end{bmatrix}
	+
	\begin{bmatrix}
		L_s & 0 & L_m & 0\\
		0 & L_s & 0 & L_m\\
		L_m & 0 & L_r & 0\\
		0 & L_m & 0 & L_r
	\end{bmatrix}
	\frac{d}{dt}
	\begin{bmatrix}
		i_{qs}\\
		i_{ds}\\
		i_{qr}\\
		i_{dr}
	\end{bmatrix}
\end{align}


\chapter{Chapter name 2}
It is induced from \eqref{eq:test}, See Page. \pageref{eq:test} \\
It is induced from \cref{eq:test}, See Page. \pageref{eq:test} \\
It is induced from \Cref{eq:test}, See Page. \pageref{eq:test} \\
It is referred from \cite{bocker2007state}


\section{Section name 2}
% It is referred from \cite{einstein1905zur}

%%%%%%%%%%%%%%%%%%%%%%%%%%%%%%%%%%%%
%% subsections
%%%%%%%%%%%%%%%%%%%%%%%%%%%%%%%%%%%%
\subsection{Subsection name}

\begin{align}
	\begin{bmatrix}
		1 & 1 \\
		0 & 0 \\
	\end{bmatrix}
\end{align}




\bibliographystyle{unsrt}
\bibliography{Latex_Guide_bibtextest}






\end{document}
